\documentclass{article}
\usepackage{amsmath}
\begin{document}


Q1\\\\
As sample size $>$ 30 central limit theory applies: \\
$\sigma_{\overline{x}} = \frac{\sigma}{\sqrt{n}} = \frac{.62}{\sqrt{35}} = .105$ \\
$\mu_{\overline{x}} = \mu = 98$

\begin{equation*}
\begin{split}
P(\overline{X} <= 98.2) & = P(Z < \frac{98.2 - \mu_{\overline{x}}}{\sigma_{\overline{x}}}) \\
& = P(Z < \frac{98.2 - 98}{.105})\\
& = P(Z < 1.9)\\
& = .86
\end{split}
\end{equation*}

\begin{equation*}
\begin{split}
P(random person) &= P( X< 98.2)\\
& = .86
\end{split}
\end{equation*}

\begin{equation*}
\begin{split}
P(97.85 <= X <= 98.15) & = P(\frac{97.85 - \mu_{\overline{x}}}{\sigma_{\overline{x}}} < Z < \frac{98.15 - \mu_{\overline{x}}}{\sigma_{\overline{x}}})\\
& = P(\frac{97.85 - 98}{.105} < Z < \frac{98.15 - 98}{.105})\\
& = P(-1.43 < Z < 1.43)\\
& = .924 - .764\\
& = 0.16
\end{split}
\end{equation*}

\begin{equation*}
\begin{split}
P(X > 98.2 ) & = P(Z > \frac{98.2 - \mu_{\overline{x}}}{\sigma_{\overline{x}}})\\
& = P(Z > \frac{98.2 - 98}{.105})\\
& = P(Z > 1.90)\\
& = 1 - P(Z < 1.90)\\
& = 1 - .971\\
& = 0.029
\end{split}
\end{equation*}

\begin{equation*}
1- pnorm(98.7, 98, .62/sqrt(35)) = 1.199241e-11
\end{equation*}
\\

Q2\\\\

\begin{equation*}
\begin{split}
P(\textrm{within 3 units}) & = P(\frac{3}{\sigma_{\overline{x}}} < Z < \frac{-3}{\sigma_{\overline{x}}})\\
& = P(\frac{3}{1.8} < Z < \frac{-3}{1.8})\\
& = P(1.67 < Z < -1.67)\\
& = 0.953 - 0.475\\
& = 0.478
\end{split}
\end{equation*}
\\

Q3\\\\
lv = 8.7 + qnorm(.1) * 2.2\\
up = 8.7 + qnorm(.9) * 2.2\\
Range is 5.9 tp 11.5 hrs/week of excercise is the average exercise of the middle 80\%\\
\\

\begin{equation*}
\begin{split}
P(Diabetic | At Risk) & = 




\end{document}