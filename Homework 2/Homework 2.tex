\documentclass{article}
\usepackage{amsmath}
\begin{document}

NOTE - TOTALLY USED CHATGPT...\\

Q1\\\\
As sample size $>$ 30 central limit theory applies: \\
$\sigma_{\overline{x}} = \frac{\sigma}{\sqrt{n}} = \frac{.62}{\sqrt{35}} = .105$ \\
$\mu_{\overline{x}} = \mu = 98$

\begin{equation*}
\begin{split}
P(\overline{X} <= 98.2) & = P(Z < \frac{98.2 - \mu_{\overline{x}}}{\sigma_{\overline{x}}}) \\
& = P(Z < \frac{98.2 - 98}{.105})\\
& = P(Z < 1.9)\\
& = .86
\end{split}
\end{equation*}

\begin{equation*}
\begin{split}
P(random person) &= P( X< 98.2)\\
& = .86
\end{split}
\end{equation*}

\begin{equation*}
\begin{split}
P(97.85 <= X <= 98.15) & = P(\frac{97.85 - \mu_{\overline{x}}}{\sigma_{\overline{x}}} < Z < \frac{98.15 - \mu_{\overline{x}}}{\sigma_{\overline{x}}})\\
& = P(\frac{97.85 - 98}{.105} < Z < \frac{98.15 - 98}{.105})\\
& = P(-1.43 < Z < 1.43)\\
& = .924 - .764\\
& = 0.16
\end{split}
\end{equation*}

\begin{equation*}
\begin{split}
P(X > 98.2 ) & = P(Z > \frac{98.2 - \mu_{\overline{x}}}{\sigma_{\overline{x}}})\\
& = P(Z > \frac{98.2 - 98}{.105})\\
& = P(Z > 1.90)\\
& = 1 - P(Z < 1.90)\\
& = 1 - .971\\
& = 0.029
\end{split}
\end{equation*}

\begin{equation*}
1- pnorm(98.7, 98, .62/sqrt(35)) = 1.199241e-11
\end{equation*}
\\

Q2\\\\

\begin{equation*}
\begin{split}
P(\textrm{within 3 units}) & = P(\frac{3}{\sigma_{\overline{x}}} < Z < \frac{-3}{\sigma_{\overline{x}}})\\
& = P(\frac{3}{1.8} < Z < \frac{-3}{1.8})\\
& = P(1.67 < Z < -1.67)\\
& = 0.953 - 0.475\\
& = 0.478
\end{split}
\end{equation*}
\\

Q3\\\\
lv = 8.7 + qnorm(.1) * 2.2\\
up = 8.7 + qnorm(.9) * 2.2\\
Range is 5.9 tp 11.5 hrs/week of excercise is the average exercise of the middle 80\%\\
\\

\begin{equation*}
\begin{split}
P(at Risk) & = P(X< 6)\\
& = P(Z <{6 - 8.6 \over 2.2} )\\
& = P(Z < -1.22)\\
& = 0.102 \\
P(Diabetic | At Risk) & = {P(At Risk | Diabetic) \over P(At Risk)} \\
& = {.078 \over .102}\\
& = .76
\end{split}
\end{equation*}
\\

Q4\\\\
I don't have the data :o\\

Q5\\\\
\begin{equation*}
\begin{split}
CI & = \hat{x} \pm Z^* {s \over \sqrt{n}}\\
& = 9.7 \pm 1.96 {4.3 \over  \sqrt{18}}\\
& = 9.7 \pm 1.99\\
& = 7.71, 11.7
\end{split}
\end{equation*}
\\

Practical Interpretation: A 95\% confidence interval for the population mean rating of acceptable load is a range of values within which we are 95\% confident that the true population mean rating of acceptable load lies. In other words, if we were to take many random samples of 18 male postal workers and calculate 95\% confidence intervals for each sample mean, we would expect approximately 95\% of those intervals to contain the true population mean rating of acceptable load.\\\\
Probability Interpretation: The probability that the true population mean rating of acceptable load falls within the calculated 95\% confidence interval is 95\%.\\\\
CI is affected by inverse square of population, so  CI range will shrink\\\\
CI range is proportional to  confidence so val increases\\\\
CI range is proportional so decrease sdev decreases CI\\\\
\\
Param of interest is mean rating of acceptable load (X)\\
Null Hypothesis: $X = 12kg$\\
Alternative Hypothesis: $X \neq 12kg$\\
Sinificance Level:$\alpha = .05$\\
Confirm data is normal\\
Can use Z (assuming data is normal)\\
\begin{equation*}
\begin{split}
z & = {\overline x - \mu \over ({\sigma \over \sqrt{n}})}\\
& = {(9.7 - 12) \over (4.3 {1\over\sqrt{18}}) }\\
& = -2.27
\end{split}
\end{equation*}
\\
For 5\% sig level critical value is 1.96. $|-2.27| > 1.96 $ so test passes\\\\
If 12kg was within our initial confidence interval
\\

Q6\\\\
\begin{equation*}
\begin{split}
4 & = Z^* {s \over \sqrt{n}}\\
n &= ({Z^x s \over 4})^2\\
& =  ({1.64 \cdot \sqrt{126} \over 4} )^2\\
& = 21 \rightarrow 22 people
\end{split}
\end{equation*}



\end{document}
